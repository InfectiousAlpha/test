\ifdefined\ispartofbook
\else
  % Preamble for "A Journey into Deep Learning"

% --- DOCUMENT CLASS & GEOMETRY ---
\documentclass[11pt, a4paper]{report} % Changed from 'article' to 'report' to enable \chapter command
\usepackage[margin=1in]{geometry} % Set page margins

% --- FONT & ENCODING ---
\usepackage[T1]{fontenc}
\usepackage[utf8]{inputenc}

% --- MATHEMATICS & SYMBOLS ---
\usepackage{amsmath, amssymb, amsthm} % For advanced math typesetting and theorems
\usepackage{amsfonts}                 % For math fonts
\usepackage{bm}                       % For bold math symbols

% --- GRAPHICS & TABLES ---
\usepackage{graphicx}                 % To include graphics
\usepackage{booktabs}                 % For professional-quality tables
\usepackage{caption}                  % For customizing captions

% --- LISTS & LAYOUT ---
\usepackage{enumitem}                 % For list customization
\usepackage{cases}                    % For cases environment

% --- TIKZ GRAPHICS PACKAGES (NEWLY ADDED) ---
\usepackage{tikz}
\usetikzlibrary{
    positioning,
    arrows.meta,
    fit,
    decorations.pathreplacing,
    calligraphy,
    shapes.geometric,
    shadows,
    chains,
    backgrounds % For layering
}

% --- HYPERLINKS & URLS ---
\usepackage[hyphens]{url}             % For URL formatting
\usepackage{hyperref}                 % For hyperlinks and cross-references
\hypersetup{
    colorlinks=true,
    linkcolor=blue,
    filecolor=magenta,
    urlcolor=cyan,
    citecolor=red,
}

% --- CUSTOM COMMANDS ---
\newcommand{\vect}[1]{\mathbf{#1}}
\newcommand{\matr}[1]{\mathbf{#1}}
\newcommand{\normdist}{\mathcal{N}}
\newcommand{\reals}{\mathbb{R}}
\newcommand{\E}{\mathbb{E}}

  \begin{document}
\fi

\chapter{Modern Frontiers and Applications}
\label{chap:frontiers}

This final chapter will look at how the foundational principles of rectification and neural networks, as detailed throughout the book, are being applied and extended at the cutting edge of machine learning research. It will also explore applications of these statistical concepts in other scientific domains, demonstrating their broad utility. This chapter serves to connect the core theory to real-world applications and state-of-the-art models, showing that the simple operation of rectification is not just a historical footnote but an active and evolving principle in the quest for more powerful and efficient AI. We will explore its role in a new class of generative models, use a practical case study to illustrate the limitations of simpler linear models, and finally, examine how the rectified multivariate normal distribution is a powerful modeling tool in its own right.

\section{Rectified Flow and Generative Modeling}
\label{sec:rectified_flow}

In recent years, the field of generative modeling has been dominated by two major paradigms: Generative Adversarial Networks (GANs) and, more recently, Denoising Diffusion Probabilistic Models (DDPMs). While incredibly powerful, diffusion models in particular have been hampered by a significant practical drawback: they require a large number of iterative steps—often hundreds or thousands—to generate a high-quality sample from noise. This slow sampling process limits their applicability in real-time or resource-constrained scenarios.

\subsection{The Principle of "Straightening" Trajectories}
\textbf{Rectified Flow} is a state-of-the-art generative modeling technique that directly addresses this limitation by elevating the concept of "rectification" to a powerful guiding principle \cite{Liu2022RectifiedFlow}. The core idea is to learn an Ordinary Differential Equation (ODE) that describes a "flow" or "transport" of probability mass from a simple source distribution (e.g., a standard Gaussian noise) to a complex target data distribution (e.g., a dataset of images).

Unlike diffusion models, which learn a complex, curved path to reverse a carefully designed noise-adding process, Rectified Flow is explicitly designed to learn a velocity field that follows the straight-line paths connecting coupled data points as much as possible. In this context, "rectified" means "straightened." The motivation for this approach is both theoretical and practical:
\begin{itemize}
    \item \textbf{Theoretical Motivation:} Straight-line paths are the shortest possible trajectories between two points in Euclidean space. A flow with straight paths represents the most efficient possible transport, minimizing the "work" required to morph the noise distribution into the data distribution.
    \item \textbf{Practical Motivation:} ODEs that describe straight-line motion can be simulated with extreme accuracy even with very coarse time discretization. A highly curved trajectory requires many small steps to approximate accurately, whereas a straight line can be traversed in a single large step. This means a high-quality sample can be generated in very few steps, sometimes even a single step, representing a massive improvement in sampling efficiency over diffusion models.
\end{itemize}

The training is achieved through a simple and scalable least-squares optimization problem. One starts by forming pairs of samples $(\vect{x}_0, \vect{x}_1)$, where $\vect{x}_0 \sim \pi_0$ (the noise distribution) and $\vect{x}_1 \sim \pi_1$ (the data distribution). A neural network is then trained to predict the vector difference $(\vect{x}_1 - \vect{x}_0)$ given an interpolated point $\vect{x}_t = (1-t)\vect{x}_0 + t \vect{x}_1$ for a randomly sampled time $t \in [0, 1]$. This procedure learns a velocity field that, on average, points along these straight paths.

\subsection{The "Reflow" Procedure and State-of-the-Art Performance}
While the first learned flow (termed "1-rectified flow") already produces paths that are straighter than those from standard diffusion models, the framework includes a powerful iterative enhancement called "reflow". The reflow procedure works as follows:
\begin{enumerate}
    \item Train an initial rectified flow model, $v_1(\vect{x}, t)$.
    \item Use this model to generate a new dataset of trajectories by simulating the flow from $\pi_0$ to $\pi_1$. This creates a new, more deterministically coupled dataset of endpoint pairs $(\vect{x}_0, \vect{x}_1')$.
    \item Train a \textit{new} rectified flow model, $v_2(\vect{x}, t)$, on this simulated data.
\end{enumerate}
This iterative process has the remarkable property of producing a sequence of flows with increasingly straight trajectories. Empirically, rectified flow models have achieved state-of-the-art results on image generation and translation tasks using as few as a single Euler discretization step, a massive improvement in sampling efficiency over previous methods.

\subsection{The Contested Link to Optimal Transport}
Given that Rectified Flow aims to find straight-line paths between distributions, it bears a superficial resemblance to the mathematical theory of \textbf{Optimal Transport (OT)}. OT is concerned with finding the most cost-efficient way to morph one distribution into another. For the specific case of the squared Euclidean cost ($L_2$), the OT map is known to be the one that transports mass along straight, parallel lines.

This led to initial claims and hope that Rectified Flow could serve as a scalable new method for solving OT problems. However, more recent and rigorous mathematical analysis has largely refuted this strong connection. The fundamental issue is a divergence in objectives:
\begin{itemize}
    \item \textbf{Rectified Flow} is designed to learn a simple (straight) and invertible map suitable for generative modeling. Its training objective minimizes the deviation from straight-line paths for an \textit{arbitrary initial coupling} of the data (e.g., random pairing).
    \item \textbf{Optimal Transport} is designed to find the \textit{specific coupling} that minimizes a global transport cost.
\end{itemize}
While RF has desirable properties, it does not guarantee that it will find the minimum-cost transport plan. In conclusion, while Rectified Flow is a superb and highly efficient framework for generative modeling, it is generally not a reliable method for computing optimal transport maps.

\section{A Case Study in Generative Modeling: The Limitations of a Single Gaussian for MNIST}
\label{sec:mnist_case_study}

A powerful pedagogical tool for understanding the limitations of simple generative models—and motivating the need for the complex, non-linear models used in modern AI—is to apply a single, high-dimensional multivariate normal (MVN) distribution to the task of image generation on the MNIST dataset \cite{LeCun1998MNIST}. This case study provides a concrete, visual demonstration of key theoretical limitations, particularly the consequences of model mismatch and the curse of dimensionality.

\subsection{Methodology: Modeling an Entire Dataset as One Gaussian}
The methodology is conceptually straightforward, relying on the principles of conditional Gaussian distributions. The MNIST dataset consists of 60,000 training images of handwritten digits, each of size 28x28 pixels.
\begin{enumerate}
    \item \textbf{Data Representation:} Each image is flattened into a single $p=784$-dimensional vector.
    \item \textbf{Model Assumption:} The entire training set of 60,000 vectors is assumed to be drawn from a single 784-dimensional MVN distribution, $\vect{x} \sim \mathcal{N}(\boldsymbol{\mu}, \boldsymbol{\Sigma})$.
    \item \textbf{Model Fitting:} The parameters of this distribution are estimated via Maximum Likelihood, which simply corresponds to calculating the sample mean vector ($\boldsymbol{\mu}$) and the sample covariance matrix ($\boldsymbol{\Sigma}$) from the training data. The covariance matrix $\boldsymbol{\Sigma}$ is a massive $784 \times 784$ matrix that captures the pairwise correlation between every pixel and every other pixel across the entire dataset. For example, $\Sigma_{ij}$ will be high if pixels $i$ and $j$ tend to be "on" or "off" together across all digits.
    \item \textbf{Generative Task (In-painting):} The generative task is image completion. Given an image with a missing portion (e.g., the bottom half of a digit), we can use the learned MVN model to predict the missing pixels. This is achieved by partitioning the 784-dimensional vector into observed pixels ($\vect{x}_b$) and unknown pixels ($\vect{x}_a$) and then computing the conditional distribution $p(\vect{x}_a | \vect{x}_b)$. The most likely completion is given by the conditional mean, $\E[\vect{x}_a | \vect{x}_b]$.
\end{enumerate}

\subsection{Analysis of Results: The Flaws of a Unimodal Model}
When this procedure is carried out, the generated image completions are coherent and respect the global structural patterns of digits. However, they suffer from two fatal flaws that perfectly illustrate the model's limitations.

\subsubsection{The Multi-Modality Problem}
The MNIST dataset is a mixture of 10 distinct classes (the digits 0 through 9) and is therefore inherently \textbf{multi-modal}. A single MVN is, by definition, \textbf{unimodal}. Fitting one MVN to all 10 classes forces the model to learn a single "average" digit that represents the central tendency of the entire dataset. The learned mean vector, $\boldsymbol{\mu}$, will look like a blurry superposition of all ten digits.

This model mismatch has a direct visual consequence: \textbf{blurriness}. When the model is asked to complete a partial digit, its prediction—the conditional mean—is an average of all possible completions over all ten digit classes, weighted by their likelihood given the observed pixels. For example, if the model is shown the top half of a digit that could be a '3', an '8', or a '5', it will not choose one. Instead, it will produce a blurry, ghostly superposition of the expected lower halves of all three digits. The model is incapable of generating sharp, class-specific features because its underlying probability distribution has only one peak, located at the "average" of all digits.

\subsubsection{The Curse of Dimensionality}
The second flaw is the method's complete lack of scalability, a direct result of the curse of dimensionality.
\begin{itemize}
    \item \textbf{Memory Complexity:} Storing the full covariance matrix requires $O(p^2)$ memory, where $p$ is the number of pixels. For MNIST ($p=784$), this is manageable: $784^2 \approx 615,000$ floating-point numbers (approx. 2.5 MB). However, for even a small 128x128 grayscale image, $p=16,384$. The covariance matrix would have over 268 million entries, requiring more than 1 GB of RAM to store. For a modest 1-megapixel image, this would require terabytes of RAM.
    \item \textbf{Time Complexity:} The true bottleneck is the matrix inversion required during the generation step to compute the conditional mean. Standard matrix inversion algorithms have a time complexity of $O(k^3)$, where $k$ is the size of the matrix being inverted (in this case, the number of observed pixels). For completing the bottom half of an MNIST image, this requires inverting a $392 \times 392$ matrix, which is fast on modern hardware. But for a 128x128 image, this could mean inverting an $8192 \times 8192$ matrix, which is computationally prohibitive.
\end{itemize}
This case study provides a clear, visual demonstration of why the field of generative modeling has moved towards sophisticated, non-linear models like Variational Autoencoders (VAEs), GANs, and Diffusion Models. These models use deep neural networks to learn complex, multi-modal data distributions in a scalable way that is not susceptible to the curse of dimensionality in the same manner.

\section{Applications in Signal Processing and Pattern Recognition}
\label{sec:signal_processing_apps}

The Rectified Multivariate Normal (RMVN) distribution is a powerful modeling tool in its own right, with significant applications in domains where non-negativity is a natural constraint or where complex, multi-modal data structures need to be represented.

\subsection{Modeling Non-Negative and Sparse Signals}
Many signals encountered in the real world are inherently non-negative, such as pixel intensities, power spectral densities, or chemical concentrations. Modeling such data with a standard Gaussian is physically unrealistic. The RMVN provides a principled and natural prior distribution for such data within a Bayesian framework.

A key application is in solving the \textbf{Sparse Non-Negative Least Squares (S-NNLS)} problem, which arises in signal recovery and compressed sensing. A powerful Bayesian approach is to place a sparsity-inducing prior on the unknown signal that also respects the non-negativity constraint. The \textbf{Rectified Gaussian Scale Mixture (R-GSM)} has emerged as a highly effective prior for this task. An R-GSM models each component of the signal as having a rectified Gaussian distribution whose variance is itself a random variable. This hierarchical model can encompass a wide range of heavy-tailed, sparsity-inducing distributions. Algorithms based on this framework, such as \textbf{Rectified Sparse Bayesian Learning (R-SBL)}, have been shown to outperform other S-NNLS solvers in both signal recovery and support identification \cite{Ding2013RectifiedFactor, Zayyani2016FastRectified}.

\subsection{Representing Complex Pattern Manifolds}
One of the most profound advantages of the RMVN over the standard MVN is its ability to be \textbf{multimodal}. A standard MVN is inherently unimodal; its probability density is maximized at the mean $\bm{\mu}$ and decreases monotonically away from it. The energy function associated with the MVN, $(\vect{x}-\bm{\mu})^T\bm{\Sigma}^{-1}(\vect{x}-\bm{\mu})$, is a convex quadratic form with a single minimum, which dictates this unimodal shape.

The RMVN, by contrast, is defined over the non-negative orthant. This constraint allows for the use of a broader class of energy functions. Specifically, the matrix in the quadratic form of the energy function need only be \textbf{copositive} (i.e., $\vect{x}^T\matr{A}\vect{x} \ge 0$ for all $\vect{x} \ge \vect{0}$), a weaker condition than the positive definiteness required for a standard MVN. This allows the energy function to be non-convex, possessing multiple local minima within the non-negative orthant. Each of these minima corresponds to a mode (a peak) in the probability distribution.

This capacity for multimodality grants the RMVN a dramatic increase in representational power. Two illustrative examples are:
\begin{itemize}
    \item \textbf{Competitive Distribution:} By designing the covariance structure to induce competition between variables (e.g., strong negative off-diagonal correlations), an RMVN can be constructed whose modes are located at the corners of the basis vectors (e.g., at $[1, 0, \dots, 0]$, $[0, 1, \dots, 0]$, etc.). This creates a distribution with multiple, well-separated peaks, ideal for modeling data that arises from distinct classes.
    \item \textbf{Cooperative Distribution:} By designing the covariance structure to encode cooperative interactions (e.g., positive correlations between "neighboring" variables), an RMVN can model distributions where the modes are closely spaced along a continuous, non-linear manifold. This is critically important for applications like invariant object recognition, where different views of an object (e.g., rotations) form a continuous manifold of patterns in a high-dimensional feature space. The RMVN can capture the structure of this entire manifold, whereas a standard MVN could only model a single prototypical view.
\end{itemize}

\ifdefined\ispartofbook
\else
  \begin{thebibliography}{99}

\bibitem{AdadiBerrada2018XAISurvey}
Adadi, A., \& Berrada, M. (2018). Peeking Inside the Black-Box: A Survey on Explainable Artificial Intelligence (XAI). \textit{IEEE Access}.

\bibitem{AlainBengio2016Probes}
Alain, G., \& Bengio, Y. (2016). Understanding intermediate layers using linear classifier probes. \textit{arXiv preprint arXiv:1610.01644}.

\bibitem{Amemiya1985Econometrics}
Amemiya, T. (1985). \textit{Advanced Econometrics}. Harvard University Press.

\bibitem{Arora2025ReLUOutputDist}
Arora, R., Basu, A., Mianjy, P., \& Mukherjee, A. (2025). On the Exact Computation of the Output Distribution of a ReLU Network. \textit{arXiv preprint arXiv:2503.22082}.

\bibitem{BarrSherrill1999TruncatedNormal}
Barr, D. R., \& Sherrill, E. T. (1999). Mean and variance of truncated normal distributions. \textit{The American Statistician, 53}(4), 357-361.

\bibitem{Bishop2006PatternRecognition}
Bishop, C. M. (2006). \textit{Pattern Recognition and Machine Learning}. Springer.

\bibitem{Bottou2018Optimization}
Bottou, L., Curtis, F. E., \& Nocedal, J. (2018). Optimization methods for large-scale machine learning. \textit{SIAM Review}, 60(2), 223-311.

\bibitem{ChoSaul2014RectifiedGaussian}
Cho, K., \& Saul, L. K. (2014). On the properties of the rectified Gaussian distribution. \textit{arXiv preprint arXiv:1406.4533}.

\bibitem{DansbeckerRELUKaggle}
Dansbecker, M. (n.d.). \textit{Rectified Linear Units (ReLU) in Deep Learning}. Kaggle. Retrieved July 23, 2025, from \url{https://www.kaggle.com/code/dansbecker/rectified-linear-units-relu-in-deep-learning}

\bibitem{Ding2013RectifiedFactor}
Ding, Z., He, Z., \& Carin, L. (2013). Rectified factor networks. In \textit{Advances in Neural Information Processing Systems 26} (pp. 1106-1114).

\bibitem{FreyHinton1999Transformations}
Frey, B. J., \& Hinton, G. E. (1999). Variational learning in nonlinear Gaussian belief networks. \textit{Neural Computation}, 11(1), 193-213.

\bibitem{Fukushima1980Neocognitron}
Fukushima, K. (1980). Neocognitron: A self-organizing neural network model for a mechanism of pattern recognition unaffected by shift in position. \textit{Biological Cybernetics}, 36(4), 193-202.

\bibitem{GalGhahramani2016DropoutBayes}
Gal, Y., \& Ghahramani, Z. (2016). Dropout as a Bayesian approximation: Representing model uncertainty in deep learning. In \textit{International conference on machine learning} (pp. 1050-1059).

\bibitem{Genz2009Computation}
Genz, A., \& Bretz, F. (2009). \textit{Computation of multivariate normal and t probabilities}. Springer Science \& Business Media.

\bibitem{Geweke1991Gibbs}
Geweke, J. (1991). Efficient simulation from the multivariate normal and Student-t distributions subject to linear constraints. In \textit{Computing Science and Statistics: Proceedings of the 23rd Symposium on the Interface} (pp. 571-578).

\bibitem{GlorotBordesBengio2011DeepSparse}
Glorot, X., Bordes, A., \& Bengio, Y. (2011). Deep sparse rectifier neural networks. In \textit{Proceedings of the fourteenth international conference on artificial intelligence and statistics} (pp. 315-323).

\bibitem{GlorotBengio2010Difficulty}
Glorot, X., \& Bengio, Y. (2010). Understanding the difficulty of training deep feedforward neural networks. In \textit{Proceedings of the thirteenth international conference on artificial intelligence and statistics} (pp. 249-256).

\bibitem{Goodfellow2016Book}
Goodfellow, I., Bengio, Y., \& Courville, A. (2016). \textit{Deep Learning}. MIT Press.

\bibitem{HaninRolnick2019ActivationPatterns}
Hanin, B., \& Rolnick, D. (2019). Deep ReLU Networks Have Surprisingly Few Activation Patterns. \textit{Advances in Neural Information Processing Systems}.

\bibitem{He2015Delving}
He, K., Zhang, X., Ren, S., \& Sun, J. (2015). Delving deep into rectifiers: Surpassing human-level performance on imagenet classification. In \textit{Proceedings of the IEEE international conference on computer vision} (pp. 1026-1034).

\bibitem{Hochreiter2001GradientFlow}
Hochreiter, S., Bengio, Y., Frasconi, P., \& Schmidhuber, J. (2001). \textit{Gradient flow in recurrent nets: the difficulty of learning long-term dependencies}. In A field guide to dynamical recurrent networks. IEEE Press.

\bibitem{Hornik1989UniversalApprox}
Hornik, K., Stinchcombe, M., \& White, H. (1989). \textit{Multilayer feedforward networks are universal approximators}. Neural Networks, 2(5), 359-366.

\bibitem{KanRobotti2017TruncatedMoments}
Kan, R., \& Robotti, C. (2017). On the moments of folded and truncated multivariate normal distributions. \textit{Journal of Computational and Graphical Statistics, 26}(4), 930-934.

\bibitem{KingmaBa2014Adam}
Kingma, D. P., \& Ba, J. (2014). Adam: A method for stochastic optimization. \textit{arXiv preprint arXiv:1412.6980}.

\bibitem{LeCun1998MNIST}
LeCun, Y., Bottou, L., Bengio, Y., \& Haffner, P. (1998). Gradient-based learning applied to document recognition. \textit{Proceedings of the IEEE}, 86(11), 2278-2324.

\bibitem{LeCun2012EfficientBackprop}
LeCun, Y., Bottou, L., Orr, G. B., \& Müller, K. R. (2012). Efficient backprop. In \textit{Neural networks: Tricks of the trade} (pp. 9-48). Springer.

\bibitem{Liu2022RectifiedFlow}
Liu, X., Gong, C., \& Liu, Q. (2022). Rectified flow: A marginal preserving approach to optimal transport. \textit{arXiv preprint arXiv:2209.14577}.

\bibitem{LundbergLee2017SHAP}
Lundberg, S. M., \& Lee, S. I. (2017). A unified approach to interpreting model predictions. In \textit{Advances in neural information processing systems} (pp. 4765-4774).

\bibitem{Mazur2015BackpropExample}
Mazur, M. (2015). A Step by Step Backpropagation Example. \textit{Matt Mazur blog}. Retrieved August 7, 2025, from \url{https://mattmazur.com/2015/03/17/a-step-by-step-backpropagation-example/}

\bibitem{Murphy2012ML}
Murphy, K. P. (2012). \textit{Machine Learning: A Probabilistic Perspective}. MIT Press.

\bibitem{Muthen1990CensoredMoments}
Muthén, B. (1990). \textit{Moments of the censored and truncated bivariate normal distribution}. British Journal of Mathematical and Statistical Psychology, 43(1), 131-143.

\bibitem{NairHinton2010ReLU}
Nair, V., \& Hinton, G. E. (2010). Rectified linear units improve restricted boltzmann machines. In \textit{Proceedings of the 27th international conference on machine learning (ICML-10)} (pp. 807-814).

\bibitem{Nielsen2015Book}
Nielsen, M. A. (2015). \textit{Neural networks and deep learning}. Determination press.

\bibitem{Pinkus1999ApproximationTheory}
Pinkus, A. (1999). \textit{Approximation theory of the MLP model in neural networks}. Acta Numerica, 8, 143-195.

\bibitem{Poole2016ExponentialExpressivity}
Poole, B., Lahiri, S., Raghu, M., Sohl-Dickstein, J., \& Ganguli, S. (2016). Exponential expressivity in deep neural networks through transient chaos. \textit{Advances in neural information processing systems}, 29.

\bibitem{Ribeiro2016LIME}
Ribeiro, M. T., Singh, S., \& Guestrin, C. (2016). "Why Should I Trust You?": Explaining the Predictions of Any Classifier. \textit{Proceedings of the 22nd ACM SIGKDD International Conference on Knowledge Discovery and Data Mining}.

\bibitem{Rudin2019StopExplaining}
Rudin, C. (2019). Stop explaining black box machine learning models for high stakes decisions and use interpretable models instead. \textit{Nature Machine Intelligence}, 1(5), 206-215.

\bibitem{Rumelhart1986Backprop}
Rumelhart, D. E., Hinton, G. E., \& Williams, R. J. (1986). Learning representations by back-propagating errors. \textit{Nature}, 323(6088), 533-536.

\bibitem{Schiendorfer2020BackpropExample}
Schiendorfer, A. (2020). A worked example of backpropagation. \textit{Connecting deep dots blog}. Retrieved August 7, 2025, from \url{https://alexander-schiendorfer.github.io/2020/02/24/a-worked-example-of-backprop.html}

\bibitem{Schoenholz2017DeepInfoProp}
Schoenholz, S. S., Gilmer, J., Ganguli, S., \& Sohl-Dickstein, J. (2017). Deep information propagation. \textit{arXiv preprint arXiv:1611.01232}.

\bibitem{Selvaraju2017GradCAM}
Selvaraju, R. R., Cogswell, M., Das, A., Vedantam, R., Parikh, D., \& Batra, D. (2017). Grad-cam: Visual explanations from deep networks via gradient-based localization. In \textit{Proceedings of the IEEE international conference on computer vision} (pp. 618-626).

\bibitem{Socci1998RectifiedGaussian}
Socci, N. D., Lee, D. D., \& Seung, H. S. (1998). The rectified gaussian distribution. In \textit{Advances in Neural Information Processing Systems} (pp. 350-356).

\bibitem{Tallis1961MGF}
Tallis, G. M. (1961). The moment generating function of the truncated multi-normal distribution. \textit{Journal of the Royal Statistical Society: Series B (Methodological)}, 23(1), 223-229.

\bibitem{Tobin1958LimitedDependent}
Tobin, J. (1958). \textit{Estimation of Relationships for Limited Dependent Variables}. Econometrica, 26(1), 24-36.

\bibitem{Wright2024AnalyticCovariance}
Wright, L., et al. (2024). An Analytic Solution to Covariance Propagation in Neural Networks. \textit{Proceedings of Machine Learning Research}.

\bibitem{Zayyani2016FastRectified}
Zayyani, H., Babaie-Zadeh, M., \& Jutten, C. (2016). Fast rectified sparse Bayesian learning. \textit{IEEE Transactions on Signal Processing}, 64(14), 3788-3799.

\end{thebibliography}

  \end{document}
\fi
